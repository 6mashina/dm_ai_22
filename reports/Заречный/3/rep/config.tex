\newcommand{\student}{Заречный А.О.}


% настройки предмета
\newcommand{\lessonName}{Интеллектуальный анализ данных}
\newcommand{\workName}{Предобучение нейронных сетей с использованием автоэнкодерного подхода}
\newcommand{\num}{3}
\newcommand{\teacher}{Крощенко А.А.}

% настройки работы
\newcommand{\purpose}{научиться осуществлять предобучение нейронных сетей с помощью автоэнкодерного подхода.}
\newcommand{\task}{ 
    \begin{itemize}
        \item Взять за основу любую сверточную или полносвязную архитектуру с количеством слоев более 3. Осуществить ее обучение (без предобучения) в соответствии с вариантом задания. Получить оценку эффективности модели, используя метрики, специфичные для решаемой задачи (например, MAPE – для регрессионной задачи или F1/Confusion matrix для классификационной).
        \item Выполнить обучение с предобучением, используя автоэнкодерный подход, алгоритм которого изложен в лекции. Условие останова (например, по количеству эпох) при обучении отдельных слоев с использованием автоэнкодера выбрать самостоятельно.
        \item Сравнить результаты, полученные при обучении с/без предобучения, сделать выводы.
        \item Оформить отчет по выполненной работе, загрузить исходный код и отчет в соответствующий репозиторий на github.
        
    \end{itemize}




    \begin{center}
        \begin{tabular}{|c|c|c|c|}
            \hline
            Вариант & Выборка                                                                                 & Тип задачи    & Целевая переменная \\
            \hline
            5       & \href{https://archive.ics.uci.edu/dataset/193/cardiotocography}{Cardiotocography} & классификация & CLASS/NSP \\
            \hline
            
        \end{tabular}
    \end{center}
}

\newcommand{\result}{

    Реализовали обучение предложенными методами. Получили следующие результаты:

    Для обучения была выбрана следующая архитектура:


    \begin{itemize}
    
        \item Входной слой - 21 нейрон;
        \item 1-ый скрытый слой - 21 нейрон, функция активации - ReLU;
        \item 2-ой скрытый слой - 42 нейрона, функция активации - ReLU;
        \item Выходной слой - 42 нейрона, функция активации - линейная.
    \end{itemize}

    
    
    
    \begin{table}[H]
        \centering
        \begin{tabular}{ll|ll|}
            \cline{3-4}
                                                                    &             & \multicolumn{2}{l|}{Вид обучения}                          \\ \cline{3-4} 
                                                                    &             & \multicolumn{1}{l|}{С предобученим} & Без предобучения     \\ \hline
            \multicolumn{1}{|l|}{\multirow{2}{*}{Слой 1}}        & Кол-во эпох & \multicolumn{1}{l|}{25}             & \multirow{2}{*}{---} \\ \cline{2-3}
            \multicolumn{1}{|l|}{}                               & Ошибка      & \multicolumn{1}{l|}{5393.6875}     &                      \\ \hline
            \multicolumn{1}{|l|}{\multirow{2}{*}{Слой 2}}        & Кол-во эпох & \multicolumn{1}{l|}{25}             & \multirow{2}{*}{---} \\ \cline{2-3}
            \multicolumn{1}{|l|}{}                               & Ошибка      & \multicolumn{1}{l|}{272.88852}      &                      \\ \hline
            \multicolumn{1}{|l|}{\multirow{2}{*}{Выходной слой}} & Кол-во эпох & \multicolumn{1}{l|}{50}             & 100                  \\ \cline{2-4} 
            \multicolumn{1}{|l|}{}                               & Ошибка      & \multicolumn{1}{l|}{1.26462}        & 1.0074               \\ \hline
        \end{tabular}
    \end{table}

    Для оценки результатов использовали матрицы запутанности:

    \begin{figure}[H]
        \centering
        \begin{tabular}{|c|c|c|c|c|c|c|c|c|c|c|}
            \hline
              & 1 & 2 & 3 & 4 & 5 & 6 & 7 & 8 & 9 & 10 \\ \hline
            1 & 48 & 20 & 1 & 0 & 0 & 0 & 0 & 0 & 0 & 6 \\ \hline
            2 & 5 & 107 & 0 & 1 & 1 & 10 & 1 & 1 & 0 & 1 \\ \hline
            3 & 10 & 2 & 0 & 0 & 0 & 0 & 0 & 0 & 0 & 0 \\ \hline
            4 & 0 & 3 & 0 & 5 & 0 & 10 & 0 & 0 & 0 & 0 \\ \hline
            5 & 2 & 7 & 0 & 0 & 0 & 0 & 0 & 0 & 0 & 0 \\ \hline
            6 & 0 & 11 & 0 & 0 & 0 & 42 & 5 & 2 & 0 & 0 \\ \hline
            7 & 0 & 11 & 0 & 1 & 0 & 21 & 18 & 0 & 0 & 1 \\ \hline
            8 & 0 & 0 & 0 & 0 & 0 & 2 & 2 & 17 & 0 & 0 \\ \hline
            9 & 0 & 0 & 0 & 0 & 0 & 0 & 0 & 0 & 2 & 8 \\ \hline
            10 & 16 & 1 & 0 & 0 & 0 & 0 & 0 & 0 & 1 & 24 \\ \hline
        \end{tabular}
        \caption{Матрица запутанности для обучения без предобучения}
    \end{figure}

    \begin{figure}[H]

        \centering
        \begin{tabular}{|c|c|c|c|c|c|c|c|c|c|c|}
        \hline
            & 1 & 2 & 3 & 4 & 5 & 6 & 7 & 8 & 9 & 10 \\ \hline
            1 & 38 & 29 & 0 & 0 & 0 & 0 & 0 & 0 & 0 & 8 \\ \hline
            2 & 15 & 96 & 0 & 0 & 0 & 12 & 2 & 1 & 0 & 1 \\ \hline
            3 & 0 & 12 & 0 & 0 & 0 & 0 & 0 & 0 & 0 & 0 \\ \hline
            4 & 0 & 6 & 0 & 0 & 0 & 12 & 0 & 0 & 0 & 0 \\ \hline
            5 & 1 & 6 & 0 & 0 & 0 & 0 & 1 & 0 & 0 & 1 \\ \hline
            6 & 0 & 17 & 0 & 0 & 0 & 39 & 1 & 3 & 0 & 0 \\ \hline
            7 & 0 & 19 & 0 & 0 & 0 & 13 & 15 & 4 & 0 & 1 \\ \hline
            8 & 0 & 2 & 0 & 0 & 0 & 1 & 1 & 17 & 0 & 0 \\ \hline
            9 & 0 & 0 & 0 & 0 & 0 & 0 & 0 & 0 & 4 & 6 \\ \hline
            10 & 14 & 2 & 0 & 0 & 0 & 0 & 0 & 0 & 2 & 24 \\ \hline
        \end{tabular}
        \caption{Матрица запутанности для обучения c предобучения}
    \end{figure}

    \begin{figure}[H]
        \centering
        \begin{tabular}{|l|l|l|}
        \cline{1-1} \cline{3-3}
        С предобучением &  & Без предобучения \\ \cline{1-1} \cline{3-3} 
        0.54695 &  & 0.61737 \\ \cline{1-1} \cline{3-3} 
        \end{tabular}

        \caption{Точность}

    \end{figure}

    Получили следующий результат --- на данном датасете точность при использовании предобучения при одинаковом количестве эпох не превышает точность при обучение без предобучения.

}
\newcommand{\conclusion}{
    научились осуществлять предобучение нейронных сетей с помощью автоэнкодерного подхода.    
}




