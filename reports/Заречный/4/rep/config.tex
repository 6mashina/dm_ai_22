\newcommand{\student}{Заречный А.О.}


% настройки предмета
\newcommand{\lessonName}{Интеллектуальный анализ данных}
\newcommand{\workName}{Предобучение нейронных сетей с использованием RBM}
\newcommand{\num}{4}
\newcommand{\teacher}{Крощенко А.А.}

% настройки работы
\newcommand{\purpose}{научиться осуществлять предобучение нейронных сетей с помощью RBM.}
\newcommand{\task}{ 
    \begin{itemize}


        \item Взять за основу нейронную сеть из лабораторной работы №3. Выполнить обучение с предобучением, используя стек ограниченных машин Больцмана (RBM – Restricted Boltzmann Machine), алгоритм которого изложен в лекции. Условие останова (например, по количеству эпох) при обучении отдельных слоев как RBM выбрать самостоятельно.
        \item Сравнить результаты, полученные при использовании различный подходов.
        \item Оформить отчет по выполненной работе, загрузить исходный код и отчет в соответствующий репозиторий на github.

    \end{itemize}




    \begin{center}
        \begin{tabular}{|c|c|c|c|}
            \hline
            Вариант & Выборка                                                                                 & Тип задачи    & Целевая переменная \\
            \hline
            5       & \href{https://archive.ics.uci.edu/dataset/193/cardiotocography}{Cardiotocography} & классификация & CLASS/NSP \\
            \hline
            
        \end{tabular}
    \end{center}
}

\newcommand{\result}{

    Реализовали обучение предложенными методами. Получили следующие результаты:

    Для обучения была выбрана следующая архитектура:


    \begin{itemize}
    
        \item Входной слой - 21 нейрон;
        \item 1-ый скрытый слой - 21 нейрон, функция активации - ReLU;
        \item 2-ой скрытый слой - 42 нейрона, функция активации - ReLU;
        \item Выходной слой - 42 нейрона, функция активации - линейная.
    \end{itemize}

    
    
    
    \begin{table}[H]
        \centering
        \begin{tabular}{ll|l|l|l|}
            \cline{3-5}
                                                                    &             & \multicolumn{3}{c|}{Вид обучения}                                                        \\ \cline{3-5} 
                                                                    &             & \multicolumn{1}{l|}{С предобученим(AE )} & С предобученим(RBM)    & Без предобучения     \\ \hline
            \multicolumn{1}{|l|}{\multirow{2}{*}{Слой 1}}        & Кол-во эпох & \multicolumn{1}{l|}{25}          & \multicolumn{1}{l|}{25}           & \multirow{2}{*}{---} \\ \cline{2-4}
            \multicolumn{1}{|l|}{}                               & Ошибка      & \multicolumn{1}{l|}{5393.6875}   & \multicolumn{1}{l|}{5787.82959}    &                      \\ \hline
            \multicolumn{1}{|l|}{\multirow{2}{*}{Слой 2}}        & Кол-во эпох & \multicolumn{1}{l|}{25}          & \multicolumn{1}{l|}{25}           & \multirow{2}{*}{---} \\ \cline{2-4}
            \multicolumn{1}{|l|}{}                               & Ошибка      & \multicolumn{1}{l|}{272.88852}   & \multicolumn{1}{l|}{5750.89941}    &                      \\ \hline
            \multicolumn{1}{|l|}{\multirow{2}{*}{Выходной слой}} & Кол-во эпох & \multicolumn{1}{l|}{50}          & \multicolumn{1}{l|}{50}           & 100                  \\ \cline{2-5} 
            \multicolumn{1}{|l|}{}                               & Ошибка      & \multicolumn{1}{l|}{1.26462}     & \multicolumn{1}{l|}{1.21925}      & 1.0074               \\ \hline
        \end{tabular}
    \end{table}

    Для оценки результатов использовали матрицы запутанности:

    \begin{figure}[H]
        \centering
        \begin{tabular}{|c|c|c|c|c|c|c|c|c|c|c|}
            \hline
              & 1 & 2 & 3 & 4 & 5 & 6 & 7 & 8 & 9 & 10 \\ \hline
            1 & 48 & 20 & 1 & 0 & 0 & 0 & 0 & 0 & 0 & 6 \\ \hline
            2 & 5 & 107 & 0 & 1 & 1 & 10 & 1 & 1 & 0 & 1 \\ \hline
            3 & 10 & 2 & 0 & 0 & 0 & 0 & 0 & 0 & 0 & 0 \\ \hline
            4 & 0 & 3 & 0 & 5 & 0 & 10 & 0 & 0 & 0 & 0 \\ \hline
            5 & 2 & 7 & 0 & 0 & 0 & 0 & 0 & 0 & 0 & 0 \\ \hline
            6 & 0 & 11 & 0 & 0 & 0 & 42 & 5 & 2 & 0 & 0 \\ \hline
            7 & 0 & 11 & 0 & 1 & 0 & 21 & 18 & 0 & 0 & 1 \\ \hline
            8 & 0 & 0 & 0 & 0 & 0 & 2 & 2 & 17 & 0 & 0 \\ \hline
            9 & 0 & 0 & 0 & 0 & 0 & 0 & 0 & 0 & 2 & 8 \\ \hline
            10 & 16 & 1 & 0 & 0 & 0 & 0 & 0 & 0 & 1 & 24 \\ \hline
        \end{tabular}
        \caption{Матрица запутанности для обучения без предобучения}
    \end{figure}

    \begin{figure}[H]

        \centering
        \begin{tabular}{|c|c|c|c|c|c|c|c|c|c|c|}
        \hline
            & 1 & 2 & 3 & 4 & 5 & 6 & 7 & 8 & 9 & 10 \\ \hline
            1 & 38 & 29 & 0 & 0 & 0 & 0 & 0 & 0 & 0 & 8 \\ \hline
            2 & 15 & 96 & 0 & 0 & 0 & 12 & 2 & 1 & 0 & 1 \\ \hline
            3 & 0 & 12 & 0 & 0 & 0 & 0 & 0 & 0 & 0 & 0 \\ \hline
            4 & 0 & 6 & 0 & 0 & 0 & 12 & 0 & 0 & 0 & 0 \\ \hline
            5 & 1 & 6 & 0 & 0 & 0 & 0 & 1 & 0 & 0 & 1 \\ \hline
            6 & 0 & 17 & 0 & 0 & 0 & 39 & 1 & 3 & 0 & 0 \\ \hline
            7 & 0 & 19 & 0 & 0 & 0 & 13 & 15 & 4 & 0 & 1 \\ \hline
            8 & 0 & 2 & 0 & 0 & 0 & 1 & 1 & 17 & 0 & 0 \\ \hline
            9 & 0 & 0 & 0 & 0 & 0 & 0 & 0 & 0 & 4 & 6 \\ \hline
            10 & 14 & 2 & 0 & 0 & 0 & 0 & 0 & 0 & 2 & 24 \\ \hline
        \end{tabular}
        \caption{Матрица запутанности для обучения c предобучением(AE)}
    \end{figure}

    \begin{figure}[H]

        \centering

        \begin{tabular}{|c|c|c|c|c|c|c|c|c|c|c|c|}
            \hline
            & 1  & 2  & 3 & 4 & 5 & 6  & 7  & 8  & 9  & 10 \\ \hline
            1 & 51 & 15 & 0 & 0 & 0 & 0  & 0  & 0  & 0  & 9  \\ \hline
            2 & 28 & 88 & 0 & 0 & 0 & 10 & 1  & 0  & 0  & 0  \\ \hline
            3 & 9  & 3  & 0 & 0 & 0 & 0  & 0  & 0  & 0  & 0  \\ \hline
            4 & 0  & 5  & 0 & 0 & 0 & 13 & 0  & 0  & 0  & 0  \\ \hline
            5 & 1  & 6  & 0 & 0 & 0 & 0  & 0  & 0  & 0  & 2  \\ \hline
            6 & 0  & 14 & 0 & 0 & 0 & 43 & 0  & 3  & 0  & 0  \\ \hline
            7 & 0  & 20 & 0 & 0 & 0 & 27 & 3  & 1  & 0  & 1  \\ \hline
            8 & 0  & 2  & 0 & 0 & 0 & 0  & 1  & 18 & 0  & 0  \\ \hline
            9 & 0  & 0  & 0 & 0 & 0 & 0  & 0  & 0  & 4  & 6  \\ \hline
        10 & 15 & 0  & 0 & 0 & 0 & 0  & 0  & 0  & 4  & 23 \\ \hline
        \end{tabular}
        \caption{Матрица запутанности для обучения c предобучением(AE)}
    \end{figure}



    \begin{figure}[H]
        \centering
        \begin{tabular}{|l|l|l|l|l|}
        \cline{1-1} \cline{3-3} \cline{5-5}
        С предобучением(AE) & & С предобучением(RBM)  &  & Без предобучения \\ \cline{1-1} \cline{3-3} \cline{5-5} 
        0.54695             & & 0.53991               &  & 0.61737          \\ \cline{1-1} \cline{3-3} \cline{5-5}
        \end{tabular}

        \caption{Точность}

    \end{figure}

    Получили следующий результат --- на данном датасете точность при использовании предобучения на RBM  при одинаковом количестве эпох не превышает точность при обучение без предобучения и с предобучением на автоэнкодере.

}
\newcommand{\conclusion}{
    научились осуществлять предобучение нейронных сетей с помощью RBM.    
}




